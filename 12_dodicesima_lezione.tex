\documentclass[letter,11pt]{article}

%layout
\usepackage[margin=2.5cm]{geometry}
\usepackage{parskip}
\usepackage{xcolor}
\usepackage{hyperref}
\usepackage{multicol}
\usepackage{multirow}
\usepackage{fancyhdr}


\renewcommand\footrulewidth{0.1pt}
\pagestyle{fancy}
\rfoot{pag. \thepage}
\fancyfoot[C]{%
  \begin{tabular}{c}
    {Dodicesima lezione} \\
  \end{tabular}}


\newcommand{\myCode}[1]{\colorbox{gray!30}{#1}}


\begin{document}

\section*{\Large{Dodicesima Lezione}}
\noindent\rule{16cm}{1pt}

\setlength{\parindent}{260pt}

\begin{multicols}{2}
\vskip 0.2in
\section*{Ci sono cose da fare ogni giorno}
\vskip 0.2in


\noindent Ci sono cose da fare ogni giorno: \\
lavarsi, studiare, giocare, \\
preparare la tavola a mezzogiorno. \\
\\
\noindent Ci sono cose da fare di notte: \\
chiudere gli occhi, dormire, \\
avere sogni da sognare, \\
orecchie per non sentire. \\
\\
\noindent Ci sono cose da non fare mai, \\
né di giorno né di notte, \\
né per mare né per terra: \\
per esempio la guerra. \\


\section*{I bravi signori}
\vskip 0.2in

\noindent Un signore di Scandicci \\
buttava le castagne \\
e mangiava i ricci. \\
\\
\noindent Un suo amico di Lastra a Signa \\
buttava i pinoli \\
e mangiava la pigna. \\
\\
\noindent Un suo cugino di Prato \\
mangiava la carta stagnola \\
e buttava il cioccolato. \\
\\
\noindent Tanta gente non lo sa \\
e dunque non se ne cruccia: \\
la vita la butta via \\
e mangia soltanto la buccia. \\

\end{multicols}

\section*{Vorrei un cane vero}
\vskip 0.2in

\noindent Marco ha sei anni, è un bambino vivace e intelligente. Ha due sorelle gemelle più piccole di lui, Maria e Gemma e due genitori che gli vogliono tanto bene. Marco adora gli animali e ama andare allo zoo con la mamma e il papà, perché in questo modo, può guardare tantissimi animali da vicino.  \\
\\
\noindent Giovanna e Luigi, i genitori di Marco, sono molto contenti della passione del figlio per gli animali e ad ogni festa, gli regalano sempre giocattoli a forma di cavallo, di giraffa e orsacchiotti e cagnolini di peluche. Marco è sempre felice quando riceve questo tipo di regalo e non desidera altro nella vita.\\
\\
\noindent Un giorno, i vicini della famiglia di Marco comprano un cucciolo di cane al negozio degli animali. Il cane è di razza labrador: è dolcissimo, ha sempre voglia di giocare e non sta un momento fermo. Ha un pelo morbidissimo, due occhi grandi e delle orecchie buffe. Il piccolo Marco quando vede il cucciolo per la prima volta è felicissimo ma, allo stesso tempo, è un po’ triste e deluso perché sa che il cagnolino non è suo e non può giocare sempre con lui.\\
\\
\noindent Da quel giorno, Marco decide che vuole un cane vero, solo per lui e non più giocattoli o peluche finti. Quindi, dice ai suoi genitori che per il prossimo Natale, desidera un cucciolo come regalo. Però, la mamma e il papà di Marco non sono d’accordo con il figlio e gli spiegano che non possono comprare un cane perché la loro casa è piccola e non hanno un giardino come i vicini. (continua...)

\vskip 0.2in
\section*{Ascoltiamo due persone parlare in italiano}
Video 1 \\
https://www.youtube.com/watch?v=tDESxMneCuM \\
\\
Video 2 \\
https://www.youtube.com/watch?v=UZaWrU7gApQ \\

\vskip 0.2in
\section*{Verbi al passato: IMPERFETTO}
\vskip 0.2in

\begin{tabular}{ |p{2cm}| p{3.5cm}| p{3.5cm}| p{3.5cm}|  }
      & Mangiare & Comprare & Dormire   \\
    \hline
    \hline
     &  &  &  \\ \hline
    Io      & {\bf mangiavo} & {\bf compravo}  & {\bf dormivo}   \\ \hline
    Tu      & {\bf mangiavi} & {\bf compravi} & {\bf dormivi} \\ \hline
    Lui/Lei & {\bf mangiava} & {\bf comprava} & {\bf dormiva}    \\ \hline
    Noi     & {\bf mangiavamo} & {\bf compravamo} & {\bf dormivamo}   \\ \hline
    Voi     & {\bf mangiavate} & {\bf compravate} & {\bf dormivate}   \\ \hline
    Loro    & {\bf mangiavano} & {\bf compravano} & {\bf dormivano}   \\ \hline
    \hline
\end{tabular}

\vskip 0.2in

\begin{multicols}{2}

\begin{itemize}
    \item Qual è l'imperfetto del verbo vedere?
    \item Qual è l'imperfetto del verbo sentire?
    \item Qual è l'imperfetto del verbo capire?
    \item Qual è l'imperfetto del verbo saltare?
    \item Qual è l'imperfetto del verbo essere?
    \item Qual è l'imperfetto del verbo avere?
    \item Qual è il passato prossimo del verbo vedere?
    \item Qual è il passato prossimo del verbo sentire?
    \item Qual è il passato prossimo del verbo capire?
    \item Qual è il passato prossimo del verbo saltare?
    \item Qual è il passato prossimo del verbo essere?
    \item Qual è il passato prossimo del verbo avere?


\end{itemize}
\end{multicols}

\vskip 0.2in
\section*{Completa le frasi usando il passato prossimo}

\begin{itemize}
    \item Ieri Maria ......... andata al mare.
    \item Giovanna, (tu) ......... parlato con Luigi?
    \item Sara e Carla ......... tornate ieri da Parigi.
    \item Loro ......... mangiato il gelato.
    \item Carlo ......... partito presto.
    \item Ragazze, a che ora (voi) ......... entrate in classe?
    \item Simona, (tu) ......... guardato quel film?

\end{itemize}

\vskip 0.2in




\end{document}
