\documentclass[letter,11pt]{article}

%layout
\usepackage[margin=2.5cm]{geometry}
\usepackage{parskip}
\usepackage{xcolor}
\usepackage{hyperref}
\usepackage{multicol}
\usepackage{multirow}


\newcommand{\myCode}[1]{\colorbox{gray!30}{#1}}


\begin{document}

\section*{\Large{Quinta Lezione}}
\noindent\rule{16cm}{1pt}

\setlength{\parindent}{260pt}

\vskip 0.2in
\section*{Costruire frasi interrogative}
\vskip 0.2in

\begin{tabular}{ |p{3cm}| p{3cm}| }

    \hline
    \hline
      &   \\ \hline
    {\bf Chi}   & {\bf Who}    \\ \hline
    {\bf Che cosa}    & {\bf What}       \\ \hline
    {\bf Quale} & {\bf Which}  \\ \hline
    {\bf Come}   & {\bf How}    \\ \hline
    {\bf Dove}    & {\bf Where}       \\ \hline
    {\bf Perchè} & {\bf Why}  \\ \hline
    {\bf Quando} & {\bf When}  \\ \hline
    \hline
\end{tabular}

\begin{multicols}{2}
\begin{itemize}
    \item Chi sei?
    \item Che cosa fai?
    \item Quale giornale vuoi?
    \item Come sta Giancarlo?
    \item Come stai?
    \item Dov'è la biblioteca?
    \item Perchè i bambini non dormono?
    \item Quando parte Pietro?



\end{itemize}
\end{multicols}
\vskip 0.2in


\vskip 0.2in
\section*{I verbi \underline{\emph{andare e fare}}}
\vskip 0.2in

\begin{tabular}{ |p{3cm}| p{3cm}| p{3cm}| }
      & Andare  & Fare  \\
    \hline
    \hline
     &  &   \\ \hline
    Io      & {\bf vado}   & {\bf faccio}    \\ \hline
    Tu      & {\bf vai}    & {\bf fai}       \\ \hline
    Lui/Lei & {\bf va}     & {\bf fa}        \\ \hline
    Noi     & {\bf andiamo} & {\bf facciamo}  \\ \hline
    Voi     & {\bf andate}  & {\bf fate}      \\ \hline
    Loro    & {\bf vanno}   & {\bf fanno}     \\ \hline
    \hline
\end{tabular}

\vskip 0.4in

\begin{multicols}{2}
\begin{itemize}
    \item Domani vado al mare con mio cugino Antonio.
    \item Dove vai in vacanza quest'estate?
    \item Di solito il mercoledì vado al cinema con i miei amici.
    \item Voi andate al mare, noi andiamo in piscina.
    \item Io faccio la spesa il giovedì.
    \item A che ora fate colazione la mattina?
    \item Oggi ho deciso di fare un dolce.
    \item Mi puoi fare un favore?


\end{itemize}
\end{multicols}
\vskip 0.2in

\section*{Conversazione 1}
\vskip 0.2in

\noindent\begin{tabular}{{ p{8.3cm} p{8.3cm} }}
    X: Buongiorno, vorrei prenotare una stanza per due persone. &  \\
     & Y: Buongiorno, certo! Che periodo? \\
     X: Dal 25 Agosto al 3 Luglio.  & \\
    & Y: Benissimo. Abbiamo disponibilità per quei giorni. \\
    X: Che cosa è in incluso nella prenotazione? & \\
    & Y: Noi offriamo la colazione. Volete aggiungere il pranzo? \\
    X: No grazie, pranziamo fuori. & \\
    & Y: Perfetto, grazie a voi. \\
\end{tabular}


\section*{Conversazione 2}
\vskip 0.2in

\noindent\begin{tabular}{{ p{8.3cm} p{8.3cm} }}
    M: Ciao Luca, come stai? &  \\
    & L: Ciao Marta. Tutto bene, e tu?\\
    M: Bene, grazie. & \\
    & L: Vorrei comprare un'auto. Tu che auto hai? \\
    M: Io ho una Hyunday Sonata. & \\
    & L: E' un'auto affidabile? \\
    M: Si è un'auto sicura, te la consiglio. & \\
    & L: Quanti chilometri ha? \\
    M: Ha 12000 chilometri. & \\
    & L: Dove l'hai comprata? \\
    M: L'ho comprata in una concessionaria a Roma. & \\
    & L: Ok. Mi puoi dare l'indirizzo di questa concessionaria? \\

\end{tabular}

\section*{Conversazione 3}
\vskip 0.2in

\noindent\begin{tabular}{{ p{8.3cm} p{8.3cm} }}
    L: Ciao Barbara, come stai?  \\
    & B: Ciao Laura, tutto bene , e tu?  \\
    L: Tutto bene grazie. Vuoi venire questa sera a cena a casa mia? & \\
    & B: Si, mi piacerebbe molto! \\
    L: Benissimo. Ti piace la lasagna? \\
    & B: Si, ovviamente, la lasagna mi piace molto. \\
    L: Bene! Allora ci vediamo a casa alle 7. & \\
    & B: Perfetto, a dopo. Ciao! \\

\end{tabular}
\vskip 0.2in


\vskip 0.2in
\section*{Passato, presente e futuro}
\vskip 0.2in

\begin{tabular}{ |p{3cm} |p{3cm}| p{3cm}| p{3cm}| }
    \hline
    \hline
    & Passato & Presente  & Futuro   \\
     & &   &   \\ \hline
     \hline
    {\bf Mangiare} & Io ho mangiato  &  Io mangio  & Io mangerò   \\
     &  &   &   \\ \hline
    {\bf Fare } &  Io ho fatto     & Io faccio   & Io farò      \\
     &  &   &   \\ \hline
    {\bf Andare} & Io sono andato  & Io vado     & Io andrò     \\
     &  &   &   \\ \hline
     {\bf Prendere} & Io ho preso     & Io prendo   & Io prenderò     \\
    &  &   &   \\ \hline
    {\bf Leggere} & Io ho letto     & Io leggo    & Io leggerò     \\
    &  &   &   \\ \hline
    \hline
\end{tabular}


\end{document}
