\documentclass[letter,11pt]{article}

%layout
\usepackage[margin=2.5cm]{geometry}
\usepackage{parskip}
\usepackage{xcolor}
\usepackage{hyperref}
\usepackage{multicol}
\usepackage{multirow}
\usepackage{fancyhdr}


\renewcommand\footrulewidth{0.1pt}
\pagestyle{fancy}
\rfoot{pag. \thepage}
\fancyfoot[C]{%
  \begin{tabular}{c}
    {Undicesima lezione} \\
  \end{tabular}}


\newcommand{\myCode}[1]{\colorbox{gray!30}{#1}}


\begin{document}

\section*{\Large{Undicesima Lezione}}
\noindent\rule{16cm}{1pt}

\setlength{\parindent}{260pt}

\vskip 0.2in
\section*{Un'estate fa}
\vskip 0.2in

\begin{multicols}{2}
\noindent Un'estate fa \\
La storia di noi due \\
era un po' come una favola \\
Ma l'estate va \\
e porta via con se \\
anche il meglio delle favole \\
\\
L'autostrada è là ma ci dividerà \\
L'autostrada della vacanza \\
segnerà la tua lontananza \\
\\
Un'estate fa non c'eri che tu \\
Ma l'estate somiglia a un gioco \\
è stupenda ma dura poco \\
poco, poco, poco \\
Un'estate fa \\
la storia di noi due \\
era un po' come una favola \\
un'estate in più \\
che ci regalerà \\
un autunno malinconico \\
\\
L'autostrada è là ma ci dividerà \\
L'autostrada della vacanza \\
segnerà la tua lontananza \\
\\
Un'estate fa non c'eri che tu \\
ma l'estate assomiglia a un gioco \\
è stupenda ma dura poco \\
poco, poco, poco \\
\\
E finisce qua la storia di noi due \\
sono cose che succedono \\
\\
L'autostrada della vacanza \\
segnerà la tua lontananza \\
Ma l'estate assomiglia a un gioco \\
è stupenda ma dura poco \\
poco, poco, poco \\
\\
E finisce qua la storia di noi due \\
due persone che si perdono \\
\\
L'autostrada è là \\
ma ci regalerà \\
un autunno malinconico \\
\end{multicols}

\section*{Mi sono innamorato di te}
\vskip 0.2in

\begin{multicols}{2}
\noindent Mi sono innamorato di te \\
Perché non avevo niente da fare \\
Il giorno volevo qualcuno da incontrare \\
La notte volevo qualcosa da sognare \\
\\
Mi sono innamorato di te\\
Perché non potevo più stare solo \\
Il giorno volevo parlare dei miei sogni \\
La notte parlare d'amore \\
\\
Ed ora che avrei mille cose da fare \\
Io sento i miei sogni svanire \\
Ma non so più pensare \\
A nient'altro che a te \\
\\
Mi sono innamorato di te \\
E adesso non so neppur io cosa fare \\
Il giorno mi pento d'averti incontrata \\
La notte ti vengo a cercare \\
\end{multicols}

\vskip 0.2in
\section*{Verbi al passato: IMPERFETTO}
\vskip 0.2in

\begin{tabular}{ |p{2cm}| p{3.5cm}| p{3.5cm}|  }
      & Essere & Avere   \\
    \hline
    \hline
     &  &   \\ \hline
    Io      & {\bf ero} & {\bf avevo}  \\ \hline
    Tu      & {\bf eri} & {\bf avevi} \\ \hline
    Lui/Lei & {\bf era} & {\bf aveva}   \\ \hline
    Noi     & {\bf eravamo} & {\bf avevamo}  \\ \hline
    Voi     & {\bf eravate} & {\bf avevate}  \\ \hline
    Loro    & {\bf erano} & {\bf avevano}   \\ \hline
    \hline
\end{tabular}

\vskip 0.2in

\noindent L’imperfetto indicativo indica la simultaneità nel passato rispetto a un momento nel passato (per questo nella tradizione grammaticale è considerato tempo relativo per eccellenza). Ciò significa che un enunciato con l’imperfetto raramente può star da solo senza riferirsi a un ancoraggio temporale, implicito o no. Si confrontino:

\noindent (1) ? due mesi fa Alessandro giocava con gli amici

\noindent (2) due mesi fa Alessandro ha giocato [o giocò] con gli amici

\noindent L’imperfetto in (1) richiede un contesto che indichi una simultaneità con un altro evento passato. In (2) la stessa frase contiene un passato prossimo (o remoto) che esprime un evento finito, perfettamente accettabile senza ulteriori riferimenti temporali. \\
\noindent La frase in (1) è dubbia in quanto contiene un imperfetto isolato, ma non è inaccettabile: diventa accettabile se inserita in un contesto adeguato, ad es., completandola con una frase temporale che delimita un momento in cui viene collocato lo svolgersi dell’imperfetto (3) o con un avverbiale temporale che indica una ripetizione dell’avvenimento stesso nell’intervallo temporale globale indicato (4):

\noindent (3) due mesi fa Alessandro giocava con gli amici quando all’improvviso gli si è avvicinata una persona sconosciuta

\noindent (4) due mesi fa Alessandro giocava con gli amici dalla mattina alla sera; ora non lo può più fare


\begin{multicols}{2}
\begin{itemize}
    \item Un anno fa avevo sempre mal di pancia.
    \item Quest'estate ero sempre in vacanza.
    \item Mentre tu compravi la pizza io ordinavo i panini.
    \item Quando mi hai chiamato ero in palestra.
    \item Ieri mattina pioveva e ho comprato un ombrello.
    \item Ieri pomeriggio ho chiamato Mario ma lui era occupato.
    \item Oggi sono arrivato in ritardo perchè c'era molto traffico.
    \item La prima volta che ho visto un delfino ero in Italia.
    \item Scusami, quando sei entrato non potevo salutarti, ero al telefono.
    \item Quando sei nato, io avevo già trent'anni.
    \item Un anno fa avevo ancora il gesso alla gamba.
    \item Mentre finivo la composizione, il mio computer si è spento
    \item Purtroppo quando siamo arrivati in aeroporto, era già troppo tardi.

\end{itemize}
\end{multicols}

\vskip 0.2in

\section*{RECAP}

\vskip 0.2in
\section*{Pronomi oggetto diretti}
\section*{I pronomi diretti sono utilizzati come complemento oggetto.}
\vskip 0.2in

\begin{tabular}{ |p{3cm}| p{2cm}| p{0.2cm}| p{2cm}| }
      & singolare  &    &   plurale  \\
    \hline
    \hline
     &  &      &  \\ \hline
    1ª persona & {\bf mi}   &   &  {\bf ci}  \\ \hline
    2ª persona & {\bf ti}   &   &  {\bf vi}  \\ \hline
    3ª persona & {\bf l', lo, la}   &   &  {\bf li, le}  \\ \hline
    \hline
\end{tabular}

\vskip 0.5in

\begin{multicols}{2}
\begin{itemize}
    \item Ho visto il concerto. / L'ho visto.
    \item Ho chiuso la porta / L'ho chiusa.
    \item Hanno portato i dolci. / Li hanno portati.
    \item Ho ascoltato la tua composizione / L'ho ascoltata
    \item Hanno mangiato il pollo. / L'hanno mangiato.
    \item Abbiamo già bevuto la limonata /   L'abbiamo già bevuta.
    \item Avete già fatto colazione. / L'avete già fatta.
    \item Mario ha rotto il vaso . / L'ha rotto Mario.



\end{itemize}
\end{multicols}


\vskip 0.2in

\section*{Avverbi interrogativi}
\section*{Formuliamo delle domande usando gli avverbi interrogativi.}
\vskip 0.2in

\begin{multicols}{2}
\begin{itemize}

    \item Dove? (Where)
    \item Quanto/a , Quanti/e? (How much, How many)
    \item Come? (How)
    \item Perchè? (Why)
    \item Quando? (When)
    \item Chi? (Who)
    \item Che cosa? (What)

\end{itemize}
\end{multicols}




\end{document}
