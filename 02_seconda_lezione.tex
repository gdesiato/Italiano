\documentclass[letter,11pt]{article}

%layout
\usepackage[margin=2.5cm]{geometry}
\usepackage{parskip}
\usepackage{xcolor}
\usepackage{hyperref}
\usepackage{multicol}


\newcommand{\myCode}[1]{\colorbox{gray!30}{#1}}


\begin{document}

\section*{\Large{Seconda Lezione}}
\noindent\rule{16cm}{1pt}

\section*{Verbo Essere e Avere (review)}

\section*{Articoli}


\begin{tabular}{ |p{2cm}| p{0.2cm}| p{2cm}| p{2cm}| p{0.2cm}| p{2cm}| p{2cm}| }
     &  & {\bf maschile} & & & & \\
    \hline
     aricolo preceduto da &  & singolare & plurale & & singolare & plurale \\
    \hline
    \hline
     &  &  & & & & \\ \hline
    consonante &  & {\bf il} & {\bf i} & & {\bf la}  & {\bf le} \\ \hline
    vocale & & {\bf l'} & {\bf gli} & & {\bf l'}  & {\bf le} \\ \hline
    s + consonante, gn, pn, ps, x, y, z & & {\bf lo} & {\bf gli} &   \\
    \hline
    \hline
\end{tabular}

\vskip 0.5in
\begin{multicols}{2}
\begin{itemize}
    \item Io sono il professore
    \item tu hai l'orologio rosso
    \item loro hanno il Covid
    \item Io ho la macchina nera
    \item l'albero è alto
    \item la scarpa è bianca
    \item gli alberi sono verdi
    \item le ore sono ventiquattro
    \item lo studente è giovane
    \item gli studenti sono giovani
    \item lo zaino è pesante
    \item gli zaini sono pesanti

\end{itemize}
\end{multicols}

\vskip 0.1in
\myCode{inserisci l'articolo determinativo singolare: }

\begin{multicols}{2}
\begin{enumerate}
    \item .... matita
    \item .... libro
    \item .... studente
    \item .... zaino
    \item .... albero
    \item .... università
    \item .... casa
    \item .... macchina
\end{enumerate}
\end{multicols}

\vskip 0.1in
\myCode{inserisci l'articolo determinativo plurale: }

\begin{multicols}{2}
\begin{enumerate}
    \item .... alberi
    \item .... matite
    \item .... zaini
    \item .... libri
    \item .... case
    \item .... macchine
    \item .... gatti
    \item .... cani
\end{enumerate}
\end{multicols}

\vskip 0.5in

\section*{Numeri}
\vskip 0.2in

\begin{multicols}{4}
\begin{tabular}{ |p{0.5cm}| p{2cm}| p{0.5cm}| p{3cm}| }

    \hline
    \hline

    0 & zero  \\ \hline
    1 & uno \\ \hline
    2 & due  \\ \hline
    3 & tre \\ \hline
    4 & quattro \\ \hline
    5 & cinque \\ \hline
    6 & sei \\ \hline
    7 & sette \\ \hline
    8 & otto \\ \hline
    9 & nove \\ \hline
    10 & dieci \\ \hline
    \hline
    \end{tabular}

    \begin{tabular}{ |p{0.5cm}| p{2cm}| p{0.5cm}| p{3cm}| }

    \hline
    \hline

     &   \\ \hline
    11 & undici \\ \hline
    12 & dodici  \\ \hline
    13 & tredici \\ \hline
    14 & quattordici \\ \hline
    15 & quindici \\ \hline
    16 & sedici \\ \hline
    17 & diciassette \\ \hline
    18 & diciotto \\ \hline
    19 & diciannove \\ \hline
    20 & venti \\ \hline
    \hline
    \end{tabular}

    \begin{tabular}{ |p{0.5cm}| p{2cm}| p{0.5cm}| p{3cm}| }

    \hline
    \hline

     &   \\ \hline
    21 & ventuno \\ \hline
    22 & ventidue  \\ \hline
    23 & ventitre \\ \hline
    24 & ventiquattro \\ \hline
    25 & venticinque \\ \hline
    26 & ventisei \\ \hline
    27 & ventisette \\ \hline
    28 & ventotto \\ \hline
    29 & ventinove \\ \hline
    30 & trenta \\ \hline
    \hline
    \end{tabular}

    \begin{tabular}{ |p{0.6cm}| p{2cm}| p{0.5cm}| p{3cm}| }

    \hline
    \hline

     &   \\ \hline
    40 & quaranta \\ \hline
    50 & cinquanta  \\ \hline
    60 & sessanta \\ \hline
    70 & settanta \\ \hline
    80 & ottanta \\ \hline
    90 & novanta \\ \hline
    100 & cento \\ \hline
    1000 & mille \\ \hline

    \hline
    \end{tabular}

    \end{multicols}

\vskip 0.2in
\begin{multicols}{4}
\begin{itemize}
    \item 30
    \item 38
    \item 28
    \item 21
    \item 31
    \item 134
    \item 15
    \item 46
    \item 53
    \item 84
    \item 12
    \item 47
    \item 52
    \item 92
    \item 99
    \item 77
    \item 11
    \item 73
    \item 71
    \item 88

\end{itemize}
\end{multicols}

\vskip 0.2in
\section*{Giorni della settimana}
\vskip 0.2in

\begin{tabular}{ |p{2cm}| p{2cm}| p{2cm}| p{2cm}| p{2cm}| p{2cm}| p{2cm}| }

    \hline
    \hline
    Lunedì & Martedì & Mercoledì & Giovedì & Venerdì & Sabato & Domenica\\ \hline
    \hline
\end{tabular}

\vskip 0.2in

Che giorno è oggi?
\begin{multicols}{2}
\begin{itemize}
    \item Oggi è sabato
    \item Io lavoro tutti i lunedì
    \item Tu leggi il mercoledî
    \item Noi mangiamo la pasta il venerdì
    \item Tu perndi la macchina il martedì
    \item La domenica (voi) scrivete
    \item il sabato (noi) dormiamo
    \item il giovedì ho sempre fame
\end{itemize}
\end{multicols}

\vskip 0.2in

\section*{Parti del giorno}
\vskip 0.2in

\begin{tabular}{ |p{2.5cm}| p{0.5cm}| p{2.5cm}| p{0.5cm}| p{2.5cm}| p{0.5cm}| p{2.5cm}| }

    \hline
    La mattina &  & Il pomeriggio &  & La sera &  & La notte\\ \hline
    \hline
\end{tabular}
\vskip 0.2in
Buon giorno, buona sera, buona notte.
\begin{multicols}{2}
\begin{itemize}
    \item Adesso è sabato pomerggio
    \item La notte ho sete
    \item Il pomeriggio ho sempre fame
    \item La mattina (io) leggo
    \item Tutte le notti (io) leggo
    \item Tutti i lunedì mattina (io) mangio i cereali

    \item Ci vediamo martedì sera
    \item Ci vediamo domani
    \item la mattina bevo il caffè
    \item la notte ho freddo
    \item il mercoledì pomeriggio leggo un libro
    \item Tutti i sabato sera mangio la pizza
\end{itemize}
\end{multicols}


\end{document}
