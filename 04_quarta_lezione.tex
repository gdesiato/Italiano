\documentclass[letter,11pt]{article}

%layout
\usepackage[margin=2.5cm]{geometry}
\usepackage{parskip}
\usepackage{xcolor}
\usepackage{hyperref}
\usepackage{multicol}
\usepackage{multirow}


\newcommand{\myCode}[1]{\colorbox{gray!30}{#1}}


\begin{document}

\section*{\Large{Quarta Lezione}}
\noindent\rule{16cm}{1pt}

\setlength{\parindent}{260pt}


\section*{Conversazione 1}
\vskip 0.2in

\noindent\begin{tabular}{{ p{8.3cm} p{8.3cm} }}
    J: Buonasera, piacere di conoscerti. Mi chiamo John. &  \\
     & M: Ciao John, molto piacere. Io mi chiamo Mario\\
     J: Da dove vieni? & \\
    & M: Vengo da Boston, e tu? \\
    J: Io vengo da Roma. & \\
    & M: Che lavoro fai John? \\
    J: Faccio l'avvocato. E tu? & \\
    & M: Io sono un architetto. \\
\end{tabular}



\section*{Conversazione 2}
\vskip 0.2in

\noindent\begin{tabular}{{ p{8.3cm} p{8.3cm} }}
    M: Buongiorno Giovanni. &  \\
    & G: Buongiorno Marta.\\
    M: Come stai oggi? & \\
    & G: Tutto bene, grazie, e tu? \\
    M: Bene, grazie! & \\
    & G: Dove stai andando? \\
    M: Sto andando a fare una passeggiata vicino \\
al fiume. & \\
    & G: Tu, invece, che fai qui? \\
    M: Io aspetto l'autobus. & \\
    & G: Va bene, allora ci vediamo presto. \\
    M: Si, a presto. Ciao! & \\
    & G: Ciao. \\

\end{tabular}

\section*{Conversazione 3}
\vskip 0.2in

\noindent\begin{tabular}{{ p{8.3cm} p{8.3cm} }}
    A: Cosa vuoi mangiare questa sera? &  \\
    & B: Vorrei gli gnocchi al pomodoro. Tu cosa vuoi mangiare?\\
    A: Io voglio una caprese e una bistecca. & \\
    & B: Ottima scelta. Allora prendo anche io la caprese. \\
    A: Ok, io prendo anche una Coca Cola. Tu che cosa vuoi bere? \\
    & B: Io vorrei dell'acqua frizzante con una fetta di limone. \\
    A: Bene! Allora chiamo il cameriere. & \\


\end{tabular}


\section*{Conversazione 4}
\vskip 0.2in

\noindent\begin{tabular}{{ p{8.3cm} p{8.3cm} }}
    P: Michela, quanti fratelli hai? &  \\
     & M: Ho un fratello e due sorelle.\\
    P: Quanti anni hanno? & \\
    & M: Mio fratello ha 30 anni, mia sorella ha 45 anni. \\
    P: Dove vivono? & \\
    & M: Mio fratello vive a New York, mia sorella vive a Londra. \\
    P: Che lavoro fanno? & \\
    & M: Mio fratello è un professore, mia sorella è un medico. \\


\end{tabular}


\vskip 0.5in
\section*{Altri esempi sull'utilizzo di aggettivi e pronomi possessivi:}


\vskip 0.5in
\begin{multicols}{2}
\begin{itemize}
    \item La mia casa è rossa. La tua?
    \item Il mio nome è Marco, e il tuo?
    \item Il mio cane è più grosso del tuo.
    \item Mio figlio lavora a Milano, e il tuo?
    \item La mia penna non scrive, mi presti la tua?
    \item Non serve il tuo computer, Luca ha portato il suo.
    \item Il nostro televisore è più piccolo del suo.
    \item Il vostro gatto ha sempre fame. Il mio non mangia mai.
    \item Le nostre biciclette sono nel garage. Le vostre dove sono?
    \item I miei genitori vivono a Napoli. I tuoi?
    \item I loro amici sono biondi. I nostri sono mori.
    \item Le sue scarpe sono vecchie. Le tue sono nuove.
    \item I vostri telefoni sono nuovi, invece il mio è vecchio.
    \item I loro gatti sono grassi. Il mio è magro.
    \item Tua sorella è molto bella, anche la mia.


\end{itemize}
\end{multicols}






\end{document}
