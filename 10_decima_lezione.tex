\documentclass[letter,11pt]{article}

%layout
\usepackage[margin=2.5cm]{geometry}
\usepackage{parskip}
\usepackage{xcolor}
\usepackage{hyperref}
\usepackage{multicol}
\usepackage{multirow}
\usepackage{fancyhdr}


\renewcommand\footrulewidth{0.1pt}
\pagestyle{fancy}
\rfoot{pag. \thepage}
\fancyfoot[C]{%
  \begin{tabular}{c}
    {Decima lezione} \\
  \end{tabular}}


\newcommand{\myCode}[1]{\colorbox{gray!30}{#1}}


\begin{document}

\section*{\Large{Decima Lezione}}
\noindent\rule{16cm}{1pt}

\setlength{\parindent}{260pt}

\vskip 0.2in
\section*{L'autunno  d'oro}
\vskip 0.2in

\noindent C’era una volta una bambina che viveva in una grande città con pochi alberi e non aveva mai visto l’autunno d’oro della campagna. \\
Quando ne sentiva parlare, domandava a suo padre: \\
– Ma è proprio d’oro? \\
– D’oro, d’oro – rispondeva il padre. \\
E la bambina pensava: “Una volta andrò dove c’è l’autunno d’oro; prenderò un po’ di quell’oro e mi comprerò 365 bambole, una per ogni giorno dell’anno”. \\
Finalmente una domenica suo padre la portò nei boschi. La bambina guardava incantata gli alberi dorati. Per tutta la giornata camminò nel bosco d’oro, giocando con le foglie, i funghi e gli scoiattoli. Era così contenta che si dimenticò delle bambole, perché ogni singola foglia le pareva più bella di tutte le bambole della Terra.

\vskip 0.2in
\section*{Biancaneve}
\vskip 0.2in
\noindent Tanto tempo fa, c’era una regina che cuciva seduta accanto alla cornice della finestra. Mentre guardava la neve, si punse un dito e tre gocce di sangue caddero sulla neve. \\
- Vorrei tanto avere una bambina bianca con la pelle bianca come la neve, le guance rosse come il sangue e i capelli neri come il legno di questa finestra - pensò la regina, che non aveva mai avuto una figlia. \\
Poco tempo dopo, le nacque una bambina dalla pelle bianca come la neve, dalle guance rosse come il sangue e dai capelli neri come il legno della finestra, ma la regina morì subito dopo averla data alla luce. Passò un anno e il re si sposò di nuovo, con una donna molto bella e molto orgogliosa. Ogni mattina chiedeva al suo specchio fatato: \\
- Specchio, specchio in questo castello, hai forse visto qualcuno più bello?

\vskip 0.2in
\section*{Verbi al passato: verbo essere + participio passato}
\vskip 0.2in

\begin{tabular}{ |p{2cm}| p{3.5cm}| p{3.5cm}| p{3.5cm}| }
      & Arrivare & Nascere & Venire  \\
    \hline
    \hline
     &  &  &  \\ \hline
    Io      & {\bf sono arrivato / a} & {\bf sono nato / a} & {\bf sono venuto / a}  \\ \hline
    Tu      & {\bf sei arrivato / a} & {\bf sei nato / a} & {\bf sei venuto / a}   \\ \hline
    Lui/Lei & {\bf è arrivato / a} & {\bf è nato / a} & {\bf è venuto / a}    \\ \hline
    Noi     & {\bf siamo arrivati / e} & {\bf siamo nati / e} & {\bf siamo venuti / e} \\ \hline
    Voi     & {\bf siete arrivati / e} & {\bf siete nati / e} & {\bf siete venuti / e} \\ \hline
    Loro    & {\bf sono arrivati / e} & {\bf sono nati / e} & {\bf sono venuti / e}\\ \hline
    \hline
\end{tabular}

\vskip 0.2in
\section*{Verbi al passato: verbo avere + participio passato}
\vskip 0.2in

\begin{tabular}{ |p{2cm}| p{3.5cm}| p{3.5cm}| p{4 cm}| }
      & Avere & Trovare & Comprare  \\
    \hline
    \hline
     &  &  & \\ \hline
    Io      & {\bf ho avuto}      & {\bf ho trovato}     &  {\bf ho comprato} \\ \hline
    Tu      & {\bf hai avuto}     & {\bf hai trovato}  &  {\bf hai comprato}  \\ \hline
    Lui/Lei & {\bf ha avuto}      & {\bf ha trovato}   &  {\bf ha comprato}\\ \hline
    Noi     & {\bf abbiamo avuto} & {\bf abbiamo trovato} & {\bf abbiamo comprato} \\ \hline
    Voi     & {\bf avete avuto}   & {\bf avete trovato}   &  {\bf avete comprato}  \\ \hline
    Loro    & {\bf hanno avuto}   & {\bf hanno trovato}   &  {\bf hanno comprato}  \\ \hline
    \hline
\end{tabular}

\vskip 0.5in


\begin{multicols}{2}
\begin{itemize}
    \item Oggi ho avuto mal di testa.
    \item Hai avuto ospiti ieri?
    \item Il mese scorso ho trovato un serpente in giardino.
    \item Per caso, avete trovato un quaderno rosso?
    \item Hai comprato la carne? E hai trovato le mele Fuji?
    \item Ieri mattina ho comparto un nuovo ombrello.
    \item Oggi sono arrivato in ritardo in uffcio.
    \item Siamo arrivati tardi, avete già iniziato?
    \item A che ora sei sei venuto ieri? Non ti ho visto.
    \item Quando sei nato?
    \item Siamo nati tutti e tre nel mese di Agosto.
    \item Luca mi ha detto che non hai trovato niente da comprare.
    \item Appena siamo arrivati, il bambino è nato.

\end{itemize}
\end{multicols}

\vskip 0.2in

\section*{Pronomi riflessivi}
\section*{I pronomi riflessivi sono utilizzati quando il complemento oggetto di una frase è anche il soggetto.}
\vskip 0.2in

\begin{tabular}{ |p{3cm}| p{2cm}| p{0.2cm}| p{2cm}| }
      & singolare  &    &   plurale  \\
    \hline
    \hline
     &  &      &  \\ \hline
    1ª persona & {\bf mi}   &   &  {\bf ci}  \\ \hline
    2ª persona & {\bf ti}   &   &  {\bf vi}  \\ \hline
    3ª persona & {\bf si}   &   &  {\bf si}  \\ \hline
    \hline
\end{tabular}

\vskip 0.5in

\begin{multicols}{2}
\begin{itemize}
    \item Giorgio e Marco non si piacciono.
    \item Ieri mattina mi sono svegliato tardi.
    \item A che ora vi fate la doccia?
    \item Mario si allena molto in palestra.
    \item Si sono spaventati molto.
    \item Mi piace la poesia russa.
    \item Dopo una corsa mi faccio una doccia fredda.
    \item Questa mattina mi sono preparato un panino con il prosciutto.


\end{itemize}
\end{multicols}


\vskip 0.2in
\section*{Pronomi oggetto diretti}
\section*{I pronomi diretti sono utilizzati come complemento oggetto.}
\vskip 0.2in

\begin{tabular}{ |p{3cm}| p{2cm}| p{0.2cm}| p{2cm}| }
      & singolare  &    &   plurale  \\
    \hline
    \hline
     &  &      &  \\ \hline
    1ª persona & {\bf mi}   &   &  {\bf ci}  \\ \hline
    2ª persona & {\bf ti}   &   &  {\bf vi}  \\ \hline
    3ª persona & {\bf l', lo, la}   &   &  {\bf li, le}  \\ \hline
    \hline
\end{tabular}

\vskip 0.5in

\begin{multicols}{2}
\begin{itemize}
    \item Hanno visitato il museo. / L'hanno visitato.
    \item Ho aperto il regalo / L'ho aperto.
    \item Hanno visto i bambini. / Li hanno visti.
    \item Hai visto le anatre? / Le hai viste?
    \item Hai comprato le scarpe? / Le hai comprate?
    \item Abbiamo visitato la chiesa. L'abbiamo visitata.
    \item Ho mangiato il panino. / L'ho mangiato.
    \item Mario mi ha chiamato. / Mi ha chiamato
    \item Luca ci ha bagnato. / Ci ha bagnato.
    \item Marta vi ha visto in piscina. / Vi ha visto in piscina.


\end{itemize}
\end{multicols}


\vskip 0.2in

\section*{Avverbi interrogativi}
\section*{Formuliamo delle domande usando gli avverbi interrogativi.}
\vskip 0.2in

\begin{multicols}{2}
\begin{itemize}

    \item Dove? (Where)
    \item Quanto/a , Quanti/e? (How much, How many)
    \item Come? (How)
    \item Perchè? (Why)
    \item Quando? (When)
    \item Chi? (Who)
    \item Che cosa? (What)

\end{itemize}
\end{multicols}




\end{document}
