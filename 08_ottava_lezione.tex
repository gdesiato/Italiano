\documentclass[letter,11pt]{article}

%layout
\usepackage[margin=2.5cm]{geometry}
\usepackage{parskip}
\usepackage{xcolor}
\usepackage{hyperref}
\usepackage{multicol}
\usepackage{multirow}
\usepackage{fancyhdr}


\renewcommand\footrulewidth{0.1pt}
\pagestyle{fancy}
\rfoot{pag. \thepage}
\fancyfoot[C]{%
  \begin{tabular}{c}
    {Ottava lezione} \\
  \end{tabular}}


\newcommand{\myCode}[1]{\colorbox{gray!30}{#1}}


\begin{document}

\section*{\Large{Ottava Lezione}}
\noindent\rule{16cm}{1pt}

\setlength{\parindent}{260pt}

\vskip 0.2in
\section*{Verbi al passato: verbo essere + participio passato}
\vskip 0.2in

\begin{tabular}{ |p{2cm}| p{3cm}| p{3cm}| p{3cm}| }
      & Andare & Essere & Cadere  \\
    \hline
    \hline
     &  &  &  \\ \hline
    Io      & {\bf sono andato / a} & {\bf sono stato / a} & {\bf sono caduto / a}  \\ \hline
    Tu      & {\bf sei andato / a} & {\bf sei stato / a} & {\bf sei caduto / a}   \\ \hline
    Lui/Lei & {\bf è andato / a} & {\bf è stato / a} & {\bf è caduto / a}    \\ \hline
    Noi     & {\bf siamo andati} & {\bf siamo stati} & {\bf siamo caduti} \\ \hline
    Voi     & {\bf siete andati} & {\bf siete stati} & {\bf siete caduti} \\ \hline
    Loro    & {\bf sono andati} & {\bf sono stati} & {\bf sono caduti}\\ \hline
    \hline
\end{tabular}

\vskip 0.2in
\section*{Verbi al passato: verbo avere + participio passato}
\vskip 0.2in

\begin{tabular}{ |p{2cm}| p{4cm}| p{4cm}| p{4cm}| }
      & Fare  & Vedere & Sentire  \\
    \hline
    \hline
     &  &  & \\ \hline
    Io      & {\bf ho fatto}      & {\bf ho visto}     &  {\bf ho sentito} \\ \hline
    Tu      & {\bf hai fatto}     & {\bf hai visto}  &  {\bf hai sentito}  \\ \hline
    Lui/Lei & {\bf ha fatto}      & {\bf ha visto}   &  {\bf ha sentito}\\ \hline
    Noi     & {\bf abbiamo fatto} & {\bf abbiamo visto} & {\bf abbiamo sentito} \\ \hline
    Voi     & {\bf avete fatto}   & {\bf avete visto}   &  {\bf avete sentito}  \\ \hline
    Loro    & {\bf hanno fatto}   & {\bf hanno visto}   &  {\bf hanno sentito}  \\ \hline
    \hline
\end{tabular}

\vskip 0.5in


\begin{multicols}{2}
\begin{itemize}
    \item Ieri ho visto un uccello rosso sull'albero.
    \item Ieri ho sentito una donna parlare in francese.
    \item Sono stato in Germania solo due volte.
    \item Due mesi fa sono caduta dalla bicicletta.
    \item Lo scorso mese ho visto un elicottero volare qui vicino.
    \item L'altro giorno ho visto Marco.
    \item Mario ha fatto la pasta all'uovo.
    \item Siete andati al museo?
    \item Che cosa avete visto a Roma?
    \item Che cosa avete mangiato ieri sera?
    \item Hai visto (incontrato) qualcuno al parco?
    \item Dove hai sentito il rumore?
    \item Sei caduto?

\end{itemize}
\end{multicols}

\vskip 0.2in
\section*{Scriviamo tre frasi al passato}
\vskip 0.2in

1. \hrulefill
\vskip 0.2in
\noindent 2. \hrulefill
\vskip 0.2in
\noindent 3. \hrulefill
\vskip 0.5in

\vskip 0.2in
\section*{Trasforma le frasi al passato}
\vskip 0.2in

\begin{multicols}{2}
\begin{itemize}

    \item Il ragazzo entra in pizzeria.
    \item La ragazza entra in pizzeria.
    \item Il ragazzo prende il treno.
    \item La ragazza prende il treno.
    \item I ragazzi entrano in pizzeria.
    \item Le ragazze entrano in pizzeria.
    \item I ragazzi prendono l’autobus.
    \item Le ragazze prendono l’autobus.

\end{itemize}
\end{multicols}
\vskip 0.2in

\section*{Completa le frasi al passato}
\vskip 0.2in


\begin{itemize}

    \item Tiziano (svegliarsi) .... molto presto stamattina.
    \item I bambini (giocare) .... nel parco tutto il pomeriggio.
    \item Michele non (farsi) .... la barba oggi.
    \item Non (essere) .... io a dire quella frase!
    \item Ieri sera io (cenare) .... alle 22:00 perché (lavorare) ho lavorato fino a tardi.
    \item Anna (vestirsi) .... in modo elegante per la cena.
    \item Chi (organizzare) .... il convegno l’altro giorno?
    \item La commedia era noiosa e io (addormentarsi) ....
    \item Io (leggere) .... un bel libro di fantascienza.
    \item Chi ti (dire) .... questa cosa?
    \item Io (stare) ....  molto bene in loro compagnia.
    \item Lorella (tagliarsi) ... i capelli dal parrucchiere questa mattina.

\end{itemize}

\vskip 0.2in

\section*{Conversazione 1}
\vskip 0.2in

\noindent\begin{tabular}{{ p{8.3cm} p{8.3cm} }}
    X: Ciao Antonio, che cosa hai fatto ieri?  \\
    & Y: Ciao Giovanni, ieri sono andato al cinema?  \\
    X: Che film hai visto?  & \\
    & Y: Ho visto un film d'azione ma non ricordo il titolo? \\
    X: Ah bene. Io non sono andato più al cinema da molto tempo. & \\
    & Y: Invece, tu che cosa hai fatto ieri? \\
    X: Sono andato a fare una passeggiata vicino al fiume. & \\
    & Y: Hai fatto bene! Hai visto gli scoiattoli? \\
    X: Si ho visto moltissimi scoiattoli. & \\
\end{tabular}


\vskip 0.5in

\section*{Aggettivi interrogativi:
\underline{\emph{quanto e quale} - (how much and which)}}
\vskip 0.2in

{\bf QUANTO} (how much), ha quattro forme:
\vskip 0.2in

\noindent \begin{tabular}{ |p{3cm}| p{3cm}| p{3cm}| p{3cm}|}
    Maschile  &  &   Femminile &  \\
    singolare & plurale &   singolare & plurale  \\
    \hline
    \hline
     &  &    &  \\ \hline
    {\bf quanto }  & {\bf quanti} &    {\bf quanta} & {\bf quante}   \\ \hline
    \hline
\end{tabular}

\vskip 0.2in

\noindent {\bf QUALE} (which), ha due forme:
\vskip 0.2in

\noindent \begin{tabular}{ |p{5cm}| p{5cm}| }
    Maschile/Femminile  &  Maschile/Femminile   \\
    singolare & plurale   \\
    \hline
    \hline
     &    \\ \hline
    {\bf quale}  & {\bf quali}   \\ \hline
    \hline
\end{tabular}

\vskip 0.5in

\begin{multicols}{2}
\begin{itemize}
    \item Quanti soldi hai?
    \item Quanto tempo ci vuole?
    \item Quanti libri hai letto quest'anno?
    \item Quanti studenti ci sono in classe?
    \item Quanta pasta vuoi?
    \item Qual è la tua macchina?
    \item Quali sono le note musicali?
    \item Quale gelato vuoi?
    \item Quali sono le città più famose d'Italia?
    \item In quale città vivi?


\end{itemize}
\end{multicols}

\end{document}
