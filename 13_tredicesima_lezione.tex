\documentclass[letter,11pt]{article}

%layout
\usepackage[margin=2.5cm]{geometry}
\usepackage{parskip}
\usepackage{xcolor}
\usepackage{hyperref}
\usepackage{multicol}
\usepackage{multirow}
\usepackage{fancyhdr}


\renewcommand\footrulewidth{0.1pt}
\pagestyle{fancy}
\rfoot{pag. \thepage}
\fancyfoot[C]{%
  \begin{tabular}{c}
    {Tredicesima lezione} \\
  \end{tabular}}


\newcommand{\myCode}[1]{\colorbox{gray!30}{#1}}


\begin{document}

\section*{\Large{Tredicesima Lezione}}
\noindent\rule{16cm}{1pt}

\setlength{\parindent}{260pt}

\section*{RECAP verbo essere e avere - Presente e Passto (passato prossimo e imperfetto)}

\vskip 0.1in
\section*{Parole parole}
\vskip 0.1in
\begin{multicols}{2}

\noindent Che cosa sei, che cosa sei, che cosa sei \\
Cosa sei \\
Non cambi mai, non cambi mai, non cambi mai
Proprio mai \\
Adesso ormai ci puoi provare  \\
Chiamami tormento dai, già che ci sei \\
Caramelle non ne voglio più \\

\noindent Le rose e i violini \\
Questa sera raccontali a un'altra \\
Violini e rose li posso sentire \\
Quando la cosa mi va, se mi va \\
Quando è il momento \\
E dopo si vedrà \\

\noindent Parole, parole, parole \\
Parole parole, parole \\
Parole, parole, parole \\
Parole, parole, parole \\
Parole, parole, parole \\
Soltanto parole \\
Parole tra noi \\

\noindent Che cosa sei, che cosa sei, che cosa sei \\
Cosa sei \\
Non cambi mai, non cambi mai, non cambi mai \\
Proprio mai \\
Nessuno più ti può fermare \\
Chiamami passione dai, hai visto mai \\

\noindent Caramelle non ne voglio più \\
La luna ed i grilli \\
Normalmente mi tengono sveglia \\
Mentre io voglio dormire e sognare \\
L'uomo che a volte c'è in te, quando c'è \\
Che parla meno \\
Ma può piacere a me \\

\end{multicols}

\vskip 0.1in
\section*{Mi sei scoppiato dentro al cuore}
\vskip 0.1in
\begin{multicols}{2}

\noindent Era \\
Solamente ieri sera \\
Io parlavo con gli amici \\
Scherzavamo fra di noi  \\

\noindent E tu, e tu, e tu \\
Tu sei arrivato \\
M'hai guardato \\
E allora tutto è cambiato per me \\

\noindent Mi sei scoppiato dentro al cuore \\
All'improvviso, all'improvviso \\
Non so perché \\
Non lo so perché \\
All'improvviso, all'improvviso \\

\noindent Sarà perché m'hai guardato  \\
Come nessuno m'ha guardato mai \\
Mi sento viva \\
All'improvviso per te \\

\noindent Ora
Io non ho capito ancora \\
Non so come può finire \\
Quello che succederà \\

\noindent Ma tu, ma tu, ma tu \\
Tu l'hai capito \\
L'hai capito \\
Ho visto che eri cambiato anche tu \\

\noindent Mi sei scoppiato dentro al cuore \\
All'improvviso, all'improvviso \\
Non so perché \\
Non lo so perché \\
All'improvviso, all'improvviso \\

\noindent Sarà perché m'hai guardato \\
Come nessuno m'ha guardato mai \\
Mi sento viva \\
All'improvviso per te \\

\noindent Mi sei scoppiato dentro al cuore \\
All'improvviso, all'improvviso \\
Non so perché \\
Non lo so perché \\
All'improvviso, all'improvviso \\

\noindent Sarà perché m'hai guardato \\
Come nessuno m'ha guardato mai \\
Mi sento viva \\
All'improvviso per te \\


\end{multicols}


\section*{Vorrei un cane vero}
\vskip 0.1 in

\noindent Marco è deluso e arrabbiato; non vuole ascoltare i genitori. Decide così, di rivolgersi direttamente a Babbo Natale; prende carta e penna e inizia a scrivere: \\

\noindent “Caro Babbo Natale, solo tu mi puoi aiutare! Vorrei tanto un cane per questo Natale, ma un cagnolino vero, non un peluche! Ormai sono grande, ho quasi sette anni! I miei genitori non mi capiscono e mi trattano ancora come un bambino piccolo. Lo sai, sono molto responsabile e poi faccio sempre il bravo con le mie sorelline, anche se a volte mi fanno innervosire perchè vogliono giocare con i miei giocattoli. Se mi regali un cane, ti prometto di trattarlo sempre bene, di portarlo fuori e di difenderlo dai cani più grandi.
Grazie. Ti voglio bene. \\

\noindent p.s. come al solito, ti lascio i biscotti che ti piacciono tanto sul tavolo. Se vuoi un po’ di latte, lo trovi in frigo.” \\

\noindent Dopo qualche giorno, i genitori di Marco scoprono la letterina di Marco per Babbo Natale e leggono cosa c’è scritto. Giovanna e Luigi non sanno cosa fare; ma, alla fine, si guardano negli occhi e decidono di realizzare il sogno del figlio. Così, il giorno dopo, vanno al canile più vicino; qui, si trovano davanti a tante gabbie con molti cani e non sanno quale scegliere. Ad un certo punto, un cagnolino bianco con dei grandi occhi attira la loro attenzione: il cane li guarda, sembra sorridere ed è dolce e affettuoso. I genitori di Marco capiscono che è il cane giusto per lui e per diventare parte della loro famiglia. (continua...)

\vskip 0.2in
\section*{Ascoltiamo}


\noindent Video 1 - Ti piace la cucina italiana?\\
https://www.youtube.com/watch?v=pW8Fuo1Oh-g \\

\noindent Video 2 - L'ora di pranzo\\
https://www.youtube.com/watch?v=DweH1De5ya0 \\


\vskip 0.2in
\section*{RECAP - Aggettivi e Pronomi Possessivi}
\vskip 0.2in

\begin{tabular}{ |p{3cm}| p{2cm}| p{2cm}| p{2cm}| p{2cm}| }
    & {\bf maschile} & & {\bf femminile} &  \\
    & singolare & plurale & singolare & plurale  \\
    \hline
    \hline
    & & & & \\ \hline
    Io   &  {\bf mio} & {\bf miei} & {\bf mia} & {\bf mie}  \\ \hline
    Tu   &  {\bf tuo} & {\bf tuoi} & {\bf tua} & {\bf tue}  \\ \hline
    Lui/lei &  {\bf suo} & {\bf suoi} & {\bf sua} & {\bf sue}  \\ \hline
    Noi  &  {\bf nostro} & {\bf nostri} & {\bf nostra} & {\bf nostre}  \\ \hline
    Voi  &  {\bf vostro} & {\bf vostri} & {\bf vostra} & {\bf vostre}  \\ \hline
    Loro &  {\bf loro} & {\bf loro} & {\bf loro} & {\bf loro}  \\ \hline
    \hline

\end{tabular}

\vskip 0.5in
\begin{multicols}{2}
\begin{itemize}
    \item Il mio zaino è stato rubato.
    \item Ho comprato la mia casa nel 2018.
    \item I vostri bicchieri erano pieni, ora avete bevuto tutto.
    \item Il mio computer è da buttare, e il tuo?
    \item Il suo libro ha avuto molto successo.
    \item Ho regalato la mia bicicletta a tuo fratello.
    \item Il nostro primo televisore era molto grande, ora è piccolissimo.
    \item Il loro gatto aveva sempre fame, ora non mangia più.
    \item Le nostre biciclette erano bianche, adesso sono diventate grigie.
    \item Le tue sorelle viveano a Roma, adesso vivono a Milano.
    \item I miei migliori amici hanno comprato questo libro. Te lo consiglio.
    \item Ieri le sue scarpe erano lucidissime, oggi sono opache.
    \item Marco e Luca hanno comprato i loro vestiti al mercato di Roma.


\end{itemize}
\end{multicols}


\vskip 0.2in
\section*{Completa le frasi usando il passato prossimo}

\begin{itemize}
    \item Antonio (avere) ......... mangiato un panino.
    \item Marta e Giulia (essere) ......... state in biblioteca tutto il giorno.
    \item Voi (avere) ......... camminato dall'università a casa.
    \item Gianni (avere) ......... invitato tutti i suoi amici al compleanno.
    \item I ragazzi (sono) ......... andati al cinema.
    \item Che cosa (avere) ........ fatto ieri?
    \item Chi (avere) ........ incontrato al mare?


\end{itemize}

\vskip 0.2in




\end{document}
