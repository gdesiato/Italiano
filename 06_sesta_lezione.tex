\documentclass[letter,11pt]{article}

%layout
\usepackage[margin=2.5cm]{geometry}
\usepackage{parskip}
\usepackage{xcolor}
\usepackage{hyperref}
\usepackage{multicol}
\usepackage{multirow}
\usepackage{fancyhdr}


\renewcommand\footrulewidth{0.1pt}
\pagestyle{fancy}
\rfoot{Page \thepage}
\fancyfoot[C]{%
  \begin{tabular}{c}
    {Sesta lezione} \\
  \end{tabular}}


\newcommand{\myCode}[1]{\colorbox{gray!30}{#1}}


\begin{document}

\section*{\Large{Sesta Lezione}}
\noindent\rule{16cm}{1pt}

\setlength{\parindent}{260pt}

\vskip 0.2in
\section*{Costruire frasi interrogative - parte 2}
\vskip 0.2in

\begin{tabular}{ |p{3cm}| p{3cm}| }

    \hline
    \hline
      &   \\ \hline
    {\bf Chi}   & {\bf Who}    \\ \hline
    {\bf Che cosa}    & {\bf What}       \\ \hline
    {\bf Quale} & {\bf Which}  \\ \hline
    {\bf Come}   & {\bf How}    \\ \hline
    {\bf Dove}    & {\bf Where}       \\ \hline
    {\bf Perchè} & {\bf Why}  \\ \hline
    {\bf Quando} & {\bf When}  \\ \hline
    \hline
\end{tabular}

\vskip 0.5in

\begin{multicols}{2}
\begin{itemize}
    \item Chi è l'attore nel film Titanic?
    \item Chi sono quei ragazzi?
    \item Che cosa fai questa sera?
    \item Quale pasta preferisci?
    \item Come si chiama tuo figlio?
    \item Come si chiama tua sorella?
    \item Dov'è la piscina?
    \item Perchè devi andare via?
    \item Quando andiamo in vacanza?
    \item Dove andate in vacanza quest'estate?


\end{itemize}
\end{multicols}
\vskip 0.2in


\section*{Aggettivi e pronomi dimostrativi:
\underline{\emph{questo e quello} - (this and that)}}
\vskip 0.2in

\begin{tabular}{ |p{3cm}| p{3cm}| p{1cm}| p{3cm}| p{3cm}|}
    Maschile  &  &  & Femminile &  \\
    singolare & plurale &  & singolare & plurale  \\
    \hline
    \hline
     &  &  &  &  \\ \hline
    {\bf questo, quest'}  & {\bf questi} &   & {\bf questa, quest'} & {\bf queste}   \\ \hline
    &  &  &   &  \\ \hline
    {\bf quello, quell', quel}  & {\bf quegli, quei, (quelli)} &   & {\bf quella, quell'} & {\bf quelle}   \\ \hline
    \hline
\end{tabular}

\vskip 0.4in

\begin{multicols}{2}
\begin{itemize}
    \item Quest'anno ho lavorato molto.
    \item Quest'automobile è affidabile.
    \item Questa mattina ho fatto un dolce.
    \item Questi giorni non mi sento molto bene.
    \item Quelle mele sul tavolo sono buone.
    \item Quell'albero ha più di cent'anni.
    \item Quelle montagne sembrano altissime.
    \item Quella signora è molto elegante.
    \item Quegli uomini sono pericolosi.
    \item Quell'aquila vola basso.


\end{itemize}
\end{multicols}
\vskip 0.2in

\section*{Conversazione 1}
\vskip 0.2in

\noindent\begin{tabular}{{ p{8.3cm} p{8.3cm} }}
    X: Salve. &  \\
    & Y: Salve come posso aiutarla?  \\
    X: Vorrei comprare due etti di prosciutto.  & \\
    & Y: Bene, che tipo di prosciutto vuole? \\
    X: Vorrei il San Daniele, Grazie. & \\
    & Y: Ottima scelta, vuole provare anche il prosciutto di Parma? \\
    X: Si grazie, prendo un etto anche di quello. & \\
    & Y: Benissimo. \\
\end{tabular}


\section*{Conversazione 2}
\vskip 0.2in

\noindent\begin{tabular}{{ p{8.3cm} p{8.3cm} }}
    P: Ciao Francesco, come stai? &  \\
    & F: Ciao Paolo. Tutto bene grazie, e tu?\\
    P: Non male, grazie. & \\
    & F: Hai visto l'ultimo film di Tarantino al cinema?  \\
    P: No, non l'ho visto. Chi sono gli attori protagonisti?  & \\
    & F: Lo devi vedere! Gli attori protagonisti sono DiCaprio e Pitt.  \\
    P: Di cosa parla questo film? & \\
    & F: non te lo descrivo così lo vedi.  \\
    P: Va bene. & \\

\end{tabular}

\section*{Conversazione 3}
\vskip 0.2in

\noindent\begin{tabular}{{ p{8.3cm} p{8.3cm} }}
    A: Ciao Carla, come stai?  \\
    & C: Ciao Anna, oggi non mi sento bene, ho un forte mal di testa.  \\
    A: Mi dispiace molto, hai preso qualche medicina? & \\
    & C: No, il medico mi ha consigliato di prendere una pasticca di ibuprofene, se ho bisogno. \\
    A: Ah va bene. Anche io prendo l'ibuprofene quando ho mal di testa.  \\
    & C: pensi che lo posso comprare al supermercato o solo in farmacia? \\
    A: si, sono sicura che lo puoi comprare anche al supermercato. & \\
    & C: grazie, allora vado a vedere! \\

\end{tabular}
\vskip 0.2in

\section*{Pronomi riflessivi}
\vskip 0.2in

\begin{tabular}{ |p{3cm}| p{2cm}| p{0.2cm}| p{2cm}| }
      & singolare  &    &   plurale  \\
    \hline
    \hline
     &  &      &  \\ \hline
    1ª persona & {\bf mi}   &   &  {\bf ci}  \\ \hline
    2ª persona & {\bf ti}   &   &  {\bf vi}  \\ \hline
    3ª persona & {\bf si}   &   &  {\bf si}  \\ \hline
    \hline
    \end{tabular}

\vskip 0.2in
    \noindent I pronomi riflessivi sono utilizzati nelle coniugazioni dei verbi quando la persona che compie l’azione e quella che la subisce coincidono. La forma del verbo, pertanto, diventa riflessiva:
    \vskip 0.2in
\noindent alzare (forma base) → alzarsi (forma riflessiva)\\


\begin{multicols}{2}
\begin{itemize}

    \item {\bf Mi} chiamo Giovanni.
    \item Va bene se {\bf ti} chiamo domani mattina?
    \item Quando {\bf mi} alzo dal divano {\bf mi} gira la testa.
    \item Io {\bf mi} lavo le mani.
    \item Lava{\bf ti} bene le mani.
    \item Maria {\bf si} pettina i capelli.
    \item {\bf Ci} vediamo mercoledì pomeriggio a casa di Luca.
    \item A che ora {\bf ti} sei svegliato?
    \item Marco e Livia {\bf si} amano e {\bf si} sposeranno presto.
    \item Chiara {\bf si} veste spesso di rosso.


\end{itemize}
\end{multicols}

\vskip 0.2in
\section*{Verbi che hanno solo la forma riflessiva}
\vskip 0.2in

\begin{itemize}
\item pentirsi
\item vergognarsi
\item arrabbiarsi
\item ribellarsi
\item arrendersi
\item impadronirsi
\item imbattersi
\item suicidarsi

\end{itemize}
\end{document}
