\documentclass[letter,11pt]{article}

%layout
\usepackage[margin=2.5cm]{geometry}
\usepackage{parskip}
\usepackage{xcolor}
\usepackage{hyperref}
\usepackage{multicol}
\usepackage{multirow}


\newcommand{\myCode}[1]{\colorbox{gray!30}{#1}}


\begin{document}

\section*{\Large{Terza Lezione}}
\noindent\rule{16cm}{1pt}

\section*{Gli articoli indeterminativi}
\vskip 0.2in

\begin{tabular}{ |p{4cm}| p{3cm}| p{0.2cm}| p{4cm}| p{3cm}| }
     &   {\bf maschile} & & & {\bf femminile} \\
     &  & & &\\
    \hline
     &  singolare & & & singolare  \\
    \hline
    \hline
    before &   & & before & \\ \hline
    any vowel or consonant and most groups of consonants & {\bf un} & & any vowel &  {\bf un'}  \\ \hline
    S + another consonant, GN, PN, PS, X, Y, Z & {\bf uno} & & any consonant or group of consonants &  {\bf una}  \\ \hline
    \hline

\end{tabular}

\vskip 0.5in
\begin{multicols}{2}
\begin{itemize}
    \item Io sono un professore dell'Università
    \item Tu hai un orologio d'oro
    \item Loro hanno un telefono
    \item Lei ha una macchina bianca e una macchina rossa
    \item L'albero ha un ramo
    \item Luca mangia una pizza margherita
    \item Carlo beve una birra
    \item Lo studente legge un libro
    \item Maria suona un'arpa celtica
    \item Voi avete un'amica alta
    \item Monica scrive un articolo
    \item Voi avete una bella casa
    \item Tu compri uno zaino
    \item Giuseppe è uno studente


\end{itemize}
\end{multicols}

\vskip 0.5in
\myCode{Inserisci l'articolo determinativo singolare: }

\begin{multicols}{2}
\begin{enumerate}
    \item .... matita
    \item .... libro
    \item .... studente
    \item .... ora
    \item .... attore
    \item .... attrice
    \item .... zaino
    \item .... albero
    \item .... ufficio
    \item .... università
    \item .... casa
    \item .... uomo
    \item .... macchina
    \item .... studente
\end{enumerate}
\end{multicols}

\vskip 0.5in
\section*{Gli aggettivi e i pronomi possessivi}

\vskip 0.2in

\begin{tabular}{ |p{3cm}| p{2cm}| p{2cm}| p{2cm}| p{2cm}| }
    & {\bf maschile} & & {\bf femminile} &  \\
    & singolare & plurale & singolare & plurale  \\
    \hline
    \hline
    & & & & \\ \hline
    Io   &  {\bf mio} & {\bf miei} & {\bf mia} & {\bf mie}  \\ \hline
    Tu   &  {\bf tuo} & {\bf tuoi} & {\bf tua} & {\bf tue}  \\ \hline
    Lui/lei &  {\bf suo} & {\bf suoi} & {\bf sua} & {\bf sue}  \\ \hline
    Noi  &  {\bf nostro} & {\bf nostri} & {\bf nostra} & {\bf nostre}  \\ \hline
    Voi  &  {\bf vostro} & {\bf vostri} & {\bf vostra} & {\bf vostre}  \\ \hline
    Loro &  {\bf loro} & {\bf loro} & {\bf loro} & {\bf loro}  \\ \hline
    \hline

\end{tabular}

\vskip 0.5in
\begin{multicols}{2}
\begin{itemize}
    \item Il mio cane è nero
    \item La mia casa è grande
    \item I vostri bicchieri sono pieni d'acqua
    \item Il tuo libro è noioso
    \item Il suo profumo è buono
    \item La tua bicicletta è bianca e verde
    \item Il nostro televisore è grande
    \item Il loro gatto ha sempre fame
    \item Le nostre biciclette sono nel garage
    \item Le tue sorelle vivono a Roma
    \item I miei migliori amici hanno trent'anni
    \item Le sue scarpe sono vecchie
    \item I vostri telefoni sono nuovi
    \item I loro vesiti sono nell'armadio


\end{itemize}
\end{multicols}

\vskip 0.2in
\section*{I verbi \underline{\emph{andare e fare}}}
\vskip 0.2in

\begin{tabular}{ |p{3cm}| p{2cm}| p{2cm}| }
      & Andare  & Fare  \\
    \hline
    \hline
     &  &   \\ \hline
    Io      & {\bf vado}   & {\bf faccio}    \\ \hline
    Tu      & {\bf vai}    & {\bf fai}       \\ \hline
    Lui/Lei & {\bf va}     & {\bf fa}        \\ \hline
    Noi     & {\bf andiamo} & {\bf facciamo}  \\ \hline
    Voi     & {\bf andate}  & {\bf fate}      \\ \hline
    Loro    & {\bf vanno}   & {\bf fanno}     \\ \hline
    \hline
\end{tabular}

\vskip 0.4in

\section*{I giorni della settimana \emph{(review)}}
\vskip 0.2in

\begin{tabular}{ |p{2cm}| p{2cm}| p{2cm}| p{2cm}| p{2cm}| p{2cm}| p{2cm}| }

    \hline
    \hline
    Lunedì & Martedì & Mercoledì & Giovedì & Venerdì & Sabato & Domenica\\ \hline
    \hline
\end{tabular}

\vskip 0.2in

\begin{multicols}{2}
\begin{itemize}
    \item Domani è domenica
    \item Domenica pomeriggio vado al mare
    \item Lunedì è il primo giorno della settimana
    \item Sabato e domenica andiamo in montagna
    \item Che giorno vai in palestra?
    \item Che giorno vai a fare la spesa?
    \item Giovedì mattina vado a scuola
    \item Mercoledì sera io e Giorgio facciamo una festa
\end{itemize}
\end{multicols}

\vskip 0.2in

\section*{I mesi}
\vskip 0.2in

\begin{tabular}{ |p{2.2cm}| p{2.2cm}| p{2.2cm}| p{2.2cm}| p{2.2cm}| p{2.2cm}| }

    \hline
    \hline
    Gennaio & Febbraio & Marzo & Aprile & Maggio & Giugno \\ \hline
    Luglio & Agosto & Settembre & Ottobre & Novembre & Dicembre \\ \hline
    \hline
\end{tabular}
\vskip 0.5in

\section*{La data}
\vskip 0.2in

Che giorno è oggi?
\begin{multicols}{2}
\begin{itemize}
    \item Oggi è sabato 16 luglio
    \item Il 2 agosto è il mio compleanno
    \item La scuola finisce il 31 maggio
    \item L'11 giugno andiamo in vacanza
    \item il 4 luglio è la festa d'Indipendenza
    \item Che giorno si festeggia il Natale?
    \item Qual è la tua data di nascita?
    \item La scuola inizia a settembre
\end{itemize}
\end{multicols}


\end{document}
