\documentclass[letter,11pt]{article}

%layout
\usepackage[margin=2.5cm]{geometry}
\usepackage{parskip}
\usepackage{xcolor}
\usepackage{hyperref}
\usepackage{multicol}
\usepackage{multirow}
\usepackage{fancyhdr}


\renewcommand\footrulewidth{0.1pt}
\pagestyle{fancy}
\rfoot{pag. \thepage}
\fancyfoot[C]{%
  \begin{tabular}{c}
    {Quattordicesima lezione} \\
  \end{tabular}}


\newcommand{\myCode}[1]{\colorbox{gray!30}{#1}}


\begin{document}

\section*{\Large{Quattordicesima Lezione}}
\noindent\rule{16cm}{1pt}

\setlength{\parindent}{260pt}

\section*{RECAP verbo essere e avere - Presente e Passto (passato prossimo e imperfetto)}

\vskip 0.1in
\section*{FUTURO SEMPLICE}
\vskip 0.1in

\begin{tabular}{ |p{2cm}| p{3cm}| p{3cm}| p{3cm}| p{3cm}| }
      & Essere & Avere & Fare & Andare  \\
    \hline
    \hline
     &  &  &  & \\ \hline
    Io      & {\bf sarò}      & {\bf avrò}     &  {\bf farò} &  {\bf andrò} \\ \hline
    Tu      & {\bf sarai}     & {\bf avrai}  &  {\bf farai} &  {\bf andrai}  \\ \hline
    Lui/Lei & {\bf sarà}      & {\bf avrà}   &  {\bf farà} &  {\bf andrà}\\ \hline
    Noi     & {\bf saremo} & {\bf avremo} & {\bf faremo} &  {\bf andremo} \\ \hline
    Voi     & {\bf sarete}   & {\bf avrete}   &  {\bf farete} &  {\bf andrete}  \\ \hline
    Loro    & {\bf saranno}   & {\bf avranno}   &  {\bf faranno} &  {\bf andranno}  \\ \hline
    \hline
\end{tabular}

\vskip 0.5in


\begin{multicols}{2}
\begin{itemize}
    \item Cosa farai domani?
    \item Fra due settimane andrò in Vermont.
    \item Fra tre giorni avrò quarantacinque anni.
    \item L' anno prossimo sposerò Maria..
    \item Da grande farò l'astronauta.
    \item Adesso andrai da lei e la  inviterai alla festa, chiaro?
    \item Ho sentito che l'aereo partirà con due ore di ritardo.
    \item Ho deciso che fra due anni andrò in italia.
    \item Domani mattina leggerò tutti i documenti con calma.
    \item Fra poco andremo a pranzo.
    \item Nel 2024 comprerò una casa in montagna.
    \item Dove sarai fra cinque anni?
    \item cosa mangeremo questa sera?

\end{itemize}
\end{multicols}


\section*{Completa le frasi declinando i verbi tra parentesi}
\vskip 0.1 in

\begin{multicols}{2}
\begin{itemize}
    \item L’anno prossimo (io iniziare) | inizierò | la dieta.
    \item Quando (io essere) |  | grande (io fare) |  | l’astronauta.
    \item Se vinco alla lotteria (comprare) |  | una casa tutta per me.
    \item In settembre Luigi (iscriversi) |  | in palestra.
    \item Ti prometto che (fare) |  | tutto il possibile per aiutarti.
    \item Maria (partire) |  | tra una settimana.
    \item Leo (restare) |  | a Pisa da solo.
    \item Forse un giorno (loro – incontrarsi) |  | di nuovo.
    \item Il futuro (arrivare) |  | molto presto.

\end{itemize}
\end{multicols}


\vskip 0.2in
\section*{Ascoltiamo}


\noindent Italiano per stranieri - Guardi la TV? (B1 - con sottitoli)\\
https://www.youtube.com/watch?v=wDSefnNI2Wk \\


\vskip 0.2in
\section*{Leggi la lettera e trasforma i verbi al futuro}
\vskip 0.2in

Caro Leo,\\

\noindent ti scrivo per salutarti.\\

\noindent Tra una settimana esatta (lasciare) |lascerò| Pisa e (tornare) |  | a Madrid. Sono molto contenta perché tra poco (potere) |  | riabbracciare la mia famiglia e i miei cari amici ma so che tu mi (mancare) |  | molto. Sono stati belli i mesi che ho trascorso qui a Pisa e ho imparato molte cose.\\
Ti (scrivere) |  | spesso e (aspettare) |  | le tue risposte, sono sicura che ti (rivedere) |  | presto. Quando (volere) |  | venire in Spagna ti (ospitare) |  | molto volentieri.

\noindent Salutami tanto i tuoi genitori.\\

\noindent A presto.\\


\vskip 0.2in
\section*{Riordina le frasi}

\begin{itemize}
    \item di – fidanzata – non – con – andrà – Sandro – in – la – lui – vacanza
    \item Vacanze – tre – finiranno – purtroppo – tra – le – giorni
    \item Corso -? – il – cosa – dopo – italiano – farai – d’-
    \item Questo – pensi – Maria – a – piacerà – regalo – che – ?


\end{itemize}

\vskip 0.2in
\section*{L'anno nuovo di Gianni Rodari}
\vskip 0.2in

\begin{multicols}{2}
\noindent Indovinami, indovino, \\
tu che leggi nel destino: \\
l’anno nuovo come sarà? \\
Bello, brutto o metà e metà? \\
Trovo stampato nei miei libroni  \\
che avrà di certo quattro stagioni, \\
dodici mesi, ciascuno al suo posto, \\
un carnevale e un ferragosto, \\
e il giorno dopo il lunedì \\
sarà sempre un martedì. \\
Di più per ora scritto non trovo \\
nel destino dell’anno nuovo: \\
per il resto anche quest’anno \\
sarà come gli uomini lo faranno. \\

\end{multicols}


\end{document}
