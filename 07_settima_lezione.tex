\documentclass[letter,11pt]{article}

%layout
\usepackage[margin=2.5cm]{geometry}
\usepackage{parskip}
\usepackage{xcolor}
\usepackage{hyperref}
\usepackage{multicol}
\usepackage{multirow}
\usepackage{fancyhdr}


\renewcommand\footrulewidth{0.1pt}
\pagestyle{fancy}
\rfoot{pag. \thepage}
\fancyfoot[C]{%
  \begin{tabular}{c}
    {Settima lezione} \\
  \end{tabular}}


\newcommand{\myCode}[1]{\colorbox{gray!30}{#1}}


\begin{document}

\section*{\Large{Settima Lezione}}
\noindent\rule{16cm}{1pt}

\setlength{\parindent}{260pt}

\vskip 0.2in
\section*{Verbi al passato: verbo essere + participio passato}
\vskip 0.2in

\begin{tabular}{ |p{2cm}| p{3cm}| }
      & Andare   \\
    \hline
    \hline
     &    \\ \hline
    Io      & {\bf sono andato}     \\ \hline
    Tu      & {\bf sei andato}      \\ \hline
    Lui/Lei & {\bf è andato}        \\ \hline
    Noi     & {\bf siamo andati}    \\ \hline
    Voi     & {\bf siete andati}    \\ \hline
    Loro    & {\bf siamo andati}    \\ \hline
    \hline
\end{tabular}

\vskip 0.2in
\section*{Verbi al passato: verbo avere + participio passato}
\vskip 0.2in

\begin{tabular}{ |p{2cm}| p{4cm}| p{4cm}| p{4cm}| }
      & Fare  & Comprare & Mangiare  \\
    \hline
    \hline
     &  &  & \\ \hline
    Io      & {\bf ho fatto}      & {\bf ho comprato}     &  {\bf ho mangiato} \\ \hline
    Tu      & {\bf hai fatto}     & {\bf hai comprato}  &  {\bf hai mangiato}  \\ \hline
    Lui/Lei & {\bf ha fatto}      & {\bf ha comprato}   &  {\bf ha mangiato}\\ \hline
    Noi     & {\bf abbiamo fatto} & {\bf abbiamo comprato} & {\bf abbiamo mangiato} \\ \hline
    Voi     & {\bf avete fatto}   & {\bf avete comprato}   &  {\bf avete mangiato}  \\ \hline
    Loro    & {\bf hanno fatto}   & {\bf hanno fatto}   &  {\bf hanno mangiato}  \\ \hline
    \hline
\end{tabular}

\vskip 0.5in


\begin{multicols}{2}
\begin{itemize}
    \item Ieri ho mangiato un panino al salmone.
    \item Avete mangiato il gelato al pistacchio?
    \item Ieri siamo andati in montagna.
    \item Due giorni fa ho mangiato la pizza.
    \item Lo scorso mese ho comprato un televisore.
    \item l'altro giorno ho visto Marco.
    \item Giovedì abbiamo comprato una nuova automobile.
    \item Luigi non c'è. È andato al lavoro
    \item Che cosa avete mangiato ieri sera?
    \item Che cosa avete comprato a Roma?
    \item Che cosa hai fatto mercoledì mattina?
    \item Chi ha mangiato la torta?
    \item Chi ha mangiato la torta?
    \item Dove sono andati ieri Luca e Marco?
    \item Hai letto il libro di ...?



\end{itemize}
\end{multicols}


\section*{Conversazione 1}
\vskip 0.2in

\noindent\begin{tabular}{{ p{8.3cm} p{8.3cm} }}
    L: Ciao Marco, come stai? &  \\
    & M: Ciao Luca, tutto bene grazie, e tu?  \\
    L: Bene. Ieri ti ho chiamato al telefono, ma non mi hai risposto.  & \\
    & M: Si, scusami. Ieri sono andato al mare. \\
    L: Dove sei andato precisamente? & \\
    & M: Sono andato a Viareggio. \\
    L: Bellissimo posto. Io sono andato  a Viareggio due anni fa. Quest'anno sono andato in Sicilia. & \\
\end{tabular}
\vskip 0.2in

\section*{Conversazione 2}
\vskip 0.2in

\noindent\begin{tabular}{{ p{8.3cm} p{8.3cm} }}
    P: Ciao Franco, oggi sono felice, ho comprato un nuovo paio di scarpe. &  \\
    & F: Ciao Paolo. In che negozio le hai comprate?\\
    P: Le ho comprate alla Nike in Via del Corso, a Roma. & \\
    & F: Io invece ho comprato una nuova giacca blu.  \\
    P: Ah bene, quando l'hai comprata?   & \\
    & F: L'ho comprata a Firenze una settimana fa.  \\

\end{tabular}

\section*{Conversazione 3}
\vskip 0.2in

\noindent\begin{tabular}{{ p{8.3cm} p{8.3cm} }}
    E: Ciao Patrizia, ieri ti ho visto a cena al ristorante Da Gino.  \\
    & P: Ciao Elena, si sono andata al ristorante con la mia famiglia.  \\
    E: Conosco quel ristorante! Molto buono. Che cosa avete mangiato? & \\
    & P: Abbiamo mangiato gli gnocchi al ragù di cinghiale, e l'anatra arrosto. \\
    E: Ottima scelta. Quando sono andata io, ho mangiato le fettuccina all'amatriciana, e il pollo con le patate. &  \\
    & P: Anche tu hai scelta bene! Hai mangiato anche la verdura? \\
    E: No, non ho mangiato la verdura perchè non la digerisco. &  \\


\end{tabular}

\section*{Conversazione 4}
\vskip 0.2in

\noindent\begin{tabular}{{ p{8.3cm} p{8.3cm} }}
    M: Buonasera Giovanni, Hai fatto la spesa per la festa di Sant'Antonio?  \\
    & G: Si, ieri ho comprato le bistecche e gli spiedini.   \\
    M: Hai comprato anche la verdura? & \\
    & G: No, mi sono dimenticato. Però ho comprato il vino rosso. \\
    M: Bene, allora ci vediamo giovedì a casa tua.  \\
    & G: Bene, a giovedì. \\

\end{tabular}

\vskip 0.2in

\vskip 0.2in


\section*{Aggettivi e pronomi dimostrativi:
\underline{\emph{questo e quello} - (this and that)}}
\vskip 0.2in

\begin{tabular}{ |p{3cm}| p{3cm}| p{1cm}| p{3cm}| p{3cm}|}
    Maschile  &  &  & Femminile &  \\
    singolare & plurale &  & singolare & plurale  \\
    \hline
    \hline
     &  &  &  &  \\ \hline
    {\bf questo, quest'}  & {\bf questi} &   & {\bf questa, quest'} & {\bf queste}   \\ \hline
    &  &  &   &  \\ \hline
    {\bf quello, quell', quel}  & {\bf quegli, quei, (quelli)} &   & {\bf quella, quell'} & {\bf quelle}   \\ \hline
    \hline
\end{tabular}

\vskip 0.2in

\section*{Pronomi riflessivi}
\vskip 0.2in

\begin{tabular}{ |p{3cm}| p{2cm}| p{0.2cm}| p{2cm}| }
      & singolare  &    &   plurale  \\
    \hline
    \hline
     &  &      &  \\ \hline
    1ª persona & {\bf mi}   &   &  {\bf ci}  \\ \hline
    2ª persona & {\bf ti}   &   &  {\bf vi}  \\ \hline
    3ª persona & {\bf si}   &   &  {\bf si}  \\ \hline
    \hline
    \end{tabular}

\vskip 0.2in

\begin{multicols}{2}
\begin{itemize}
    \item Quest'albero mi piace molto.
    \item La mattina ci svegliamo tutti alle 6.
    \item Questa settima Eric si veste di blu.
    \item Luigi si diverte a giocare con la palla.
    \item Ieri ti ho comprato questa maglia.
    \item Vi piacciono queste scarpe?
    \item Ti piace questo film?
    \item Mario e Luca si divertono molto al mare.
    \item Dopo una giornata al mare ci facciamo la doccia.
    \item Questa mattina non mi sono pettinato i capelli.


\end{itemize}
\end{multicols}



\end{document}
